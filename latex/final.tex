%%目次を作る際は\verb+\tableofcontents+ と打ちます。\\
%%新しいページに区切るときは\verb+\newpage+ と打ちます
\documentclass[a4j,10.5pt]{jarticle}
\usepackage[dvipdfmx]{graphicx}
\usepackage{amssymb}
\usepackage{amsmath}
\usepackage{float}
\providecommand{\keywords}[1]{\textbf{\textbf{Keywords---}} #1}
\makeatletter
\renewcommand{\thetable}{% 表番号の付け方
	\thesection-\arabic{table}}
\@addtoreset{table}{section}
\makeatother
\makeatletter
\renewcommand{\thefigure}{% 表番号の付け方
	\thesection-\arabic{figure}}
\@addtoreset{figure}{section}
\makeatother
\begin{document}
\begin{center}
%\thispagestyle{empty}
\vspace*{1.0cm}
{\large 久保研介研究会 \ 卒業論文 \\
\vspace*{3.5 mm}
\Large 医薬品市場におけるオーソライズド・ジェネリックの参入阻止効果
}\\
\vspace*{0.85cm}
{慶應義塾大学 \ 商学部4年 \\
\vspace*{2.5mm}
\large 41815144 \  難波慶成}
\vspace*{0.8cm}
\begin{abstract}
新薬メーカーは先発医薬品の特許満了時にジェネリック医薬品への代替による減収に直面する。近年、多くの新薬メーカーがオーソライズド・ジェネリックと呼ばれる、先発医薬品の薬事承認に基づいて製造販売されるジェネリック医薬品の上市を支援することで、そのような減収の軽減を図っている。品質面や安全面などにおいて優れているオーソライズド・ジェネリックの上市はジェネリック医薬品メーカーの参入意欲を低下させる可能性があることから、公正競争上の懸念がある。一方で、オーソライズド・ジェネリックの上市が当該市場におけるジェネリック医薬品の参入に与える効果を検証した先行研究の多くでは、参入阻止効果を示唆する分析結果は得られていない。本研究では、新薬メーカーによる過去のオーソライズド・ジェネリックの投入実績が、市場をまたいでジェネリック医薬品の参入数に与える効果を、参入ゲームの枠組みの下で推定した。その結果、過去のオーソライズド・ジェネリックの投入実績がジェネリック医薬品メーカーの参入確率を低下させるという仮説を統計的に支持する推定値は得られなかった。


\end{abstract}
\vspace{0.5cm}
\keywords{ジェネリック医薬品, \ オーソライズド・ジェネリック, \ 参入ゲーム}

\end{center}

%\newpage
%\pagenumbering{roman}
%\tableofcontents
%\newpage
%\pagenumbering{arabic}
\section{はじめに}

近年、日本の医療用医薬品市場においてジェネリック医薬品のシェアが拡大している。ジェネリック医薬品とは、新薬の特許満了後に厚生労働省の承認を受けて販売される、新薬と有効成分や効能が同一の医薬品である。ジェネリック医薬品は一般的に新薬メーカーの製品(以下「先発医薬品」又は「先発品」という)と比べて低価格で販売されるため、その普及によって医療費負担額が削減されることが期待されている。厚生労働省の『令和元年度ロードマップ検証検討事業報告書』によると、日本のジェネリック医薬品の数量シェアは2014年時点では49\%であったが、2020年には70\%となっており、急速に伸びている。一方で欧米と比較すると、2020年時点で米国では95\%、ドイツでは89\%、イギリスでは80\%といずれも高水準であり、欧米諸国に遅れをとっている日本においてはジェネリック医薬品が更に普及する余地があると言えよう\footnote{ただし、一部の欧米諸国は日本と比べてジェネリック医薬品の普及が遅れている。例えば厚生労働省 (2020)では、ジェネリック医薬品の数量シェアは2020年時点でイタリアでは61\%, フランスでは75\%, スペインでは65\%と報告されている。}。そのため、高齢化に伴う国民医療費の増加が懸念される日本社会においては、今後もジェネリック医薬品の普及が進むことに期待が寄せられている。\par
他方で、近年はジェネリック医薬品の流通に関連したいくつかの問題が浮上している。中でも品質管理問題は喫緊の課題である。2020年12月に中堅ジェネリック医薬品メーカーの小林化工が販売する爪水虫などの治療薬に睡眠導入成分が混入していた問題が発覚したことを皮切りに、日医工や共和薬品工業、長生堂製薬などでも生産上の不備が明るみに出た。問題の背景には、ジェネリック医薬品メーカーによる急速な生産規模の拡大や原薬の供給を海外メーカーに依存していることがあると指摘されている\footnote{日経新聞, 2021年3月4日,「日医工 急成長のひずみ(上) 主力工場、品質おざなり」など}。\par
こうしたなかで、オーソライズド・ジェネリック (Authorized Generic; 以下「AG」という) と呼ばれる新たな形態のジェネリック医薬品に注目が集まっている。AGは、先発品の製造販売承認のもと、必要に応じて新薬メーカーの特許許諾を受けながら販売されるジェネリック医薬品である。新薬メーカーの技術をそのまま利用するため、有効成分に加えて添加物も先発品と同一である。また、多くの場合は先発品の工場で生産されるため、原薬の多くをアジアからの輸入に頼る純粋なジェネリック医薬品と比べて品質への不安が少ない。一方で、そうした品質面における優位性や一足先に市場参入できるというタイミング上の優位性から、AGが純粋なジェネリック医薬品に参入阻止効果をもたらすことが懸念されている(Reiffen and Ward 2007; FTC, 2010; Peelish 2020など)。\par
本研究ではAGの上市がジェネリック医薬品メーカーに与える参入阻止効果を検証する。先行する実証研究の多くは、ある市場におけるAGの上市が当該市場におけるその後のジェネリック医薬品の参入の有無ないし参入数に与える効果を検証しているところ、そのほとんどはAGの参入阻止効果を示す推定結果を得ていない。一方で、過去に別の市場でAGが上市された実績を見て、ジェネリック医薬品メーカーがこれから参入しようとする市場でもAGの参入が起こることを危惧し、当該市場への参入を控えるというシナリオも考えられる。本研究では後者のシナリオを基に仮説を構築し、過去の別市場におけるAG投入の実績がジェネリック医薬品メーカーの参入に与える影響を参入ゲームの枠組の下で検証する。\par
本稿の次節以降の構成は次の通りである。第2節では医薬品市場を取り巻く状況や関連する用語について概説する。第3節ではジェネリック医薬品およびAGの市場参入について分析した先行研究を概観し、本研究との関係をまとめる。第4節では分析で用いるモデルを説明し、第5節で分析に使用したデータの詳細とモデルの特定化について記述する。第6節で推定結果とその解釈を提示したのち、第7節で本研究の結論を述べる。

\section{医薬品市場に関する制度的背景}
日本で販売される医薬品は医療用医薬品と一般用医薬品の2種類に大別される\footnote{公正取引委員会 (2015)}。医療用医薬品は主に医師や歯科医師によって処方ないし指示され、薬剤師の調剤により患者に供給される医薬品であり、厚生労働省が定める薬価基準に基づいて価格 (薬価) が決められる。他方、一般用医薬品は患者が自らの判断で購入する医薬品であり、医療用医薬品のような公定価格はなく、メーカーが価格を決める。\par
医療用医薬品は先発医薬品とジェネリック医薬品の2種類に大別される。先発医薬品 (先発品) は新薬とも呼ばれ、メーカーが研究開発によって発明した新しい効能を有し、臨床試験等により、その有効性や安全性が確認され、承認された医薬品である。一方でジェネリック医薬品は後発医薬品 (後発品) とも呼ばれ、新薬の特許満了後に厚生労働省に承認を受けて販売される、当該新薬と有効成分や薬効が同一の医薬品である。先発品に比べて研究開発に必要な費用が少なく済み、複数のメーカーによって競争的に供給されることから、一般的に低価格で販売される。発売される時点で有効成分に関する基本特許は満了しているが、製剤特許等のその他の特許や新薬の承認後に追加された新たな用法特許等が残っているケースもある。\par
医療用医薬品の薬価は医療保険制度に立脚した公定価格である。日本では全ての国民が何らかの公的な医療保険に加入し、保険料を支払うことにより、全国どこの保険医療機関でも医療を受けることができる国民皆保険制度が採用されている。国民が医療保険サービスの下で処方される医療用医薬品は、厚生労働大臣が健康保険法に基づき定める薬価基準に収載されている。薬価基準は医療保険から保険医療機関等に支払われる際の医薬品の価格を定めたものであり、卸売業者等から医療機関や薬局に対する実際の販売価格 (市場実勢価格) を調査し、その結果に基づき定期的に下方改定されている。そのため、薬価は市場における競争を直接反映したものではないものの、薬価基準の改定に伴って間接的に市場の価格競争を反映している。なお、薬価基準への新規収載は先発品は年4回、ジェネリック医薬品は年2回となっている。\par
薬価の算定方法は初収載品と既収載品で異なり、初収載品の薬価算定方式は先発品とジェネリック医薬品で異なる。初収載される先発品の算定方式には、主に類似薬の薬価に加算・調整等をすることにより薬価を算定する類似薬効比較方式と、製造原価に一定のマージンを加えて薬価を算定する原価計算方式の2種類がある。初収載されるジェネリック医薬品は先発品の薬価に基づいて算定される。2016年に改革された薬価制度の下では、原則として先発品の薬価の0.5掛けで算定され、内用薬でなおかつ同時期に収載されるジェネリック医薬品が10品目を超える場合に限り先発品の薬価の0.4掛けで算定される。既収載品の薬価は原則として市場実勢価格を反映して改定される。\par
このように、医療用医薬品産業は規制産業である。政府は特許制度と薬価基準に基づいて、先発品の独占を一定期間認めつつジェネリック医薬品の供給を促進することで、先発品メーカーの研究開発インセンティブと消費者利益の間のトレードオフを調整しているのである。こうした制度の下で、医薬品メーカーは互いに競争している。そして先発品メーカーとジェネリック医薬品メーカーの間に存在する利害の衝突はしばしば競争法上の問題を引き起こす。\par
オーソライズド・ジェネリック(Authorized Generic; 以下「AG」という)と呼ばれる新たな形態の医薬品は、医薬品メーカー間の競争について議論する上で重要なトピックになっている。先述のとおり、AGは新薬の製造販売承認のもと、必要に応じて先発品メーカーの特許許諾を受けながら販売されるジェネリック医薬品であるところ、新薬メーカーの技術を利用するため、有効成分に加えて添加物も先発品と同一である。日本では2013年6月にアレルギー治療薬アレグラのAGが発売されて以来、様々なAGが発売されている。新薬の特許満了後にジェネリック医薬品との競争に晒される先発品メーカーは、AGの供給者(当該供給者がAGの製造を担っているとは限らないものの、以下では「AGメーカー」と呼ぶ)から支払われるロイヤリティによってジェネリックへの代替による収入減を軽減することができる。また、AGは先発品メーカーの製造販売承認の下で供給されるため、医薬品の安全性や有効性に関する患者や医師の懸念が少ない。さらに、AGはしばしば先発品メーカーの工場で生産されるため、原薬の多くをアジアからの輸入に頼る純粋なジェネリック医薬品と比べて品質面で優れている。こうした安全面および品質面で優れたAGの流通は、一見すると経済厚生の向上に貢献するが、一方でAGが純粋なジェネリックの参入意欲を減退させることで市場における競争度が低下し、長期的には経済厚生が毀損される可能性もある。\par
実際に、AGの是非に関する政策的な議論は既に行われている。米国では,新薬メーカーの業界団体であるPharmaceutical Research and Manufacturers of America (PhRMA)の依頼を受けたIMS Consultingが、AGが医療コストに与える影響を調査したレポートを発表したことを端緒として議論が続いている(IMS Consulting, 2006)。その後、PhRMAの調査方法や内容に対する批判を踏まえ、競争当局である連邦取引委員会 (Federal Trade Commission; FTC) がAGの影響を定量的かつ包括的に分析した報告書を発表している (FTC, 2011) 。こうした分析や調査の多くは、AGが経済厚生に与える効果を有意に示すことができていないが、Peelish (2021) はAGを通じた略奪的価格設定が実現しうることを指摘することで、新たな論点を提示している。日本では、阿部 (2021) がバイオ医薬品のAGに関する論考を発表しており、AGの薬価算定において公正取引委員会の知見を有効活用する仕組み作りを提案している 。学術的には、米国や欧州における医薬品市場のデータを使ってAGの参入阻止効果を検証した経済学研究が複数発表されているが、これらについては次節で論じる。


\section{問題意識および先行研究との関係}

企業の市場参入行動は、Bresnahan and Reiss (1991)やBerry (1992) に端を発する参入ゲームに立脚した計量経済学的手法によって描写することができる。Bresnahan and Reiss (1991) は企業間の同質性を仮定した上で、企業の期待利潤を市場規模と参入企業数に依存する可変利潤関数と固定費用の差として定義している。そして、参入企業数と市場属性の関係を順序プロビットモデルによって推定することで、企業間の競争度合いを推定できることを提示している。具体的には、地理的に隔離された個別市場において市場集中度が高くなる傾向がある小売業や専門サービス業を対象に、市場規模と参入事業者数の関係を分析している。Berry (1992) は固定費用に限定して企業間の異質性を許容した上で、期待利潤が高い企業から順に市場に参入するという仮定をおくことで、複数均衡の問題を回避しつつ企業間の異質性を考慮できる実証モデルを提示し、当該モデルを航空市場のデータによって推定している。\par
医薬品市場におけるジェネリックの参入行動を検証した研究にはScott Morton (1999) がある。Scott Morton (1999) はジェネリックメーカー間の異質性の効果を検証すべく、参入可能な市場の先発医薬品の特徴と、ジェネリックメーカーが過去に販売した医薬品の特徴とのマッチ度合いによってジェネリックメーカーの固定費用が変わるとの特定化を行った上で、米国のジェネリック医薬品市場における参入行動を2項プロビットモデルとして推定している。経済厚生への含意という観点からは、ジェネリック医薬品市場における参入の動学的側面を明示的にモデル化したReiffen and Ward (2005) の貢献が大きい。彼らは米国食品医薬品局(Food and Drug Administration; FDA)による承認を受けて参入するジェネリックメーカーの参入タイミングをモデル化し、個別市場における参入メーカー数と市場価格の関係に関する構造推定を行っている。分析の結果、参入するジェネリックメーカー数の増加が医薬品価格の低下をもたらすことが実証されており、この結果から、ジェネリックメーカーの参入が消費者余剰を向上させることが示唆される。日本の医薬品市場におけるジェネリックメーカーの参入を分析した研究は筆者の知る限りIizuka (2009) のみである。彼は医薬品の参入企業数と市場特徴の関係を誘導形モデルで分析しており、Scott Morton (1999)らが示した市場規模の効果が、薬価基準に立脚した規制の下でも見られることを確認すると同時に、院内処方のシェアや慢性疾患治療薬市場に固有の現象などについて、先行研究と異なる結果を示している。\par
AGがジェネリックメーカーの参入に与える効果を検証した実証研究の多くは、米国の医薬品市場のデータを用いている。例えば、Berndt et al. (2007) はジェネリックメーカーと新薬メーカーの間の特許訴訟という観点から、AGが参入した市場とそれ以外の市場を比較しており、AGが存在する市場においてもジェネリックメーカーが特許訴訟を行うインセンティブは損なわれていないと結論付けている。しかし、彼らの分析はデータの記述と要約に留まっており、市場規模などの参入インセンティブを規定する重要な要因を統御していないため、政策的な議論のためのエビデンスとしては不十分である。また、Reiffen and Ward (2007) はReiffen and Ward (2005) の構造モデルと先行研究の推定結果を使ってカリブレーションを行い、AGが市場均衡に与える影響をシミュレーションしている。この研究では、比較的規模の小さい市場において、AGの投入が価格を高止まりさせる効果を持つことが示されている。\par
FTC (2011) は米国においてAGの参入が活発化した2003年から2008年までのデータを用いて、AGが価格や売上といった市場成果に与える影響を検証している。同報告書は、最初に参入したジェネリックメーカーが180日間の独占権 (180-day exclusivity) を行使できる期間の内外に分けて、AGの投入がもたらす短期的効果と長期的効果を分析している。180日間の独占権とはハッチ・ワックスマン法 (Hatch-Waxman Act) に基づく米国特有の権利であり、先発品メーカーとの特許訴訟で最初に勝訴又は和解したジェネリックメーカーに付与される。具体的には、180日間の独占期間中は他のジェネリックメーカーによる簡易型の製造販売承認申請(Abbreviated New Drug Application; ANDA)が承認されない。これに対し、AGは先発品の製造販売承認申請 (New Drug Application; NDA)に基づいて承認されるため、180日間の独占期間中であっても参入することができる。この180日間の独占期間内、つまり短期においては、AGが参入することによりジェネリックメーカーが競争にさらされるため、AGが参入しない場合よりも価格が低下する傾向があることが定量的に示されている。他方、180日間の独占期間の満了後つまり長期においては、競争優位なAGの存在がジェネリックの参入を抑制し、価格が高止まりする可能性がある。\par
FTC (2011)は、AGの参入による価格の高止まり効果を示唆する分析結果を得ていない。ただし、同報告書 の定量的分析は市場価格や売上に生じる効果の検証に終始しており、AGがジェネリックメーカーの参入に関する意思決定に与える影響は明示的に検証していない。一方、同報告書の定性的調査においては、AGの参入によってジェネリックメーカーが獲得できる収益が減少することに対する懸念が、一部のジェネリックメーカーによって表明されており、AGの存在がジェネリックメーカーの期待形成に一定の影響を与えていることが覗われる。\par
米国以外の医薬品市場のデータを用いた研究としては、ドイツの医薬品市場のデータを用いて、AGの上市が同じ市場におけるジェネリックの参入に与える影響を検証したAppelt (2010)がある。この研究は、ある時点での先発品メーカーによるAGの投入が、当該メーカーが開発した同時期に特許が満了する医薬品の数と相関していることに着目し、AGの参入とジェネリックメーカーの参入の2変量を使って2変量プロビットモデル (bivariate probit model)を推定している。そうすることで、AGの参入の内生性を考慮しつつ、AGの参入がジェネリックメーカーの参入確率に与える効果を推定することが可能となっている。AGの参入を表す変数の係数推定値はいずれの場合も負であるが、2変量プロビットモデルの推定値は単変量プロビットモデルのそれに比べて絶対値が小さくなることから、AG投入ダミーの内生性から生じるバイアスの存在が示唆されている。\par
本研究の動機の一つは、先行研究においては市場をまたいだAGの効果が十分に考慮されていないのではないかという問題意識である。先行研究の多くは、ある市場におけるAGの参入が当該市場におけるジェネリックメーカーの意思決定に与える影響を検証した上で、「AGの参入がジェネリックメーカーの参入を阻害するという仮説は支持されない」との結論を下している。しかし、ジェネリックメーカーの参入行動を繰り返しゲームの枠組みに当てはめると、AGの投入という先発品メーカーの行動が、たとえ当該市場においては価格の低下や消費者余剰の向上といった望ましい成果に寄与していても、翌期以降に特許満了を迎える市場では参入インセンティブを低下させる効果を持つかもしれない。より具体的には、先発品メーカーとしてはAGがジェネリックメーカーの期待形成に与える影響を予見しつつ、AGを市場に投入することでジェネリックによる参入の抑制を図っている可能性が考えられる。このような繰り返しゲームにおけるプレイヤーの行動原理は、Kreps and Wilson (1982) などが提示する逐次均衡によって理論的に説明されており、逐次均衡の枠組みの下でAGがジェネリックメーカーの参入にもたらす負の効果はPeelish (2020)でも指摘されている。\par
本研究のもう一つの動機は、既存研究のほとんどが米国市場のデータを用いているという点である。筆者の知る限り、米国以外の医薬品市場のデータを用いてAGの効果を検証した研究はAppelt (2010) のみであり、日本のデータを用いた研究は未だ存在しない。Iizuka (2009) が指摘するように、欧州を含めた米国以外の各国の医薬品市場には多くの場合何らかの法規制が存在し、それが企業行動に少なからず影響を及ぼしている。特に薬価が当局によって規制されている日本の医薬品市場では、ジェネリックメーカーの参入行動に米国のそれとは異なる要因が働いている可能性があることから、AGの効果に関する欧米の研究成果をそのまま当てはめるのではなく、新たな分析を行う必要がある。厳しい薬価規制の下でのジェネリックメーカーの参入要因やAGの効果を検証することは、米国以外の国に適用可能なインプリケーションを得ることにも繋がるだろう。\par
こうした問題意識を踏まえ、本研究では日本の医薬品市場のデータを用い、先発品メーカーによる過去のAGの投入が市場をまたいでジェネリックメーカーの参入意欲に与える影響を検証する。日本の医薬品市場では、ハッチ・ワックスマン法に規定される米国市場とは異なり、先発品の特許が満了するまでジェネリックメーカーの参入は行われないことが一般的である。また、殆どのジェネリックメーカーが特許期間満了直後に同時に参入するという点も特徴的である。そのため、米国と比べた場合、AGの存在が当該市場内でのジェネリックメーカーの参入行動に与える影響は小さいものと予想される。こうした日本の医薬品市場の性質を活用し、市場ごとに参入企業数を算定するとともに、ジェネリックメーカーが予想する「AGが投入される可能性」およびその代理変数を定義する。そして、それらの関係を分析することで、AGがもたらすジェネリックメーカーの参入インセンティブの低下を検証する。具体的な分析手法は次節で述べる。

\section{モデル}
以上の議論を踏まえ、本節ではAGが時間を通じてジェネリックメーカーの参入インセンティブに影響を与えるというシナリオの下、先発品メーカーによる過去のAGの投入実績とジェネリックメーカーの参入に関する意思決定の関係を理論的に描写する。\par
日本の医薬品市場にAGが登場してから本稿の執筆時点までに約10年が経過しているが、ジェネリック医薬品の開発に費やされる期間が一般的に3年から5年と言われていることを踏まえると、AGとジェネリックの競争の動学的側面を長期的な視点で分析することは容易ではない。そこで、ここでは単純な2期間の動学モデルを想定し、ジェネリックメーカーは初期($t = 0$)におけるAGの参入の観察を踏まえて翌期($t = 1$)以降に参入可能な市場において獲得できる利潤に関する期待形成を行い、それに基づいて翌期の参入に関する意思決定を行うと考える。これはKim (2009) の推定モデルを2期間に縮約したモデルといえる。\par
期待利潤の特定化にあたってはReiffen and Ward (2005) のモデルを参考にする。ただし、本研究の関心は先発品メーカーとジェネリックメーカーの間の競争にあるので、ジェネリックメーカー間の競争における動学的側面は捨象する。なお、最初に特許訴訟において勝訴又は和解したジェネリックメーカーに180日間の独占権が与えられる米国とは異なり、日本では全てのジェネリックの承認がほぼ同時に行われるため、この想定はある程度現実的である。一方で、1企業が獲得する利潤は当該市場における競争度に依存することを捉えるため、Bresnahan and Reiss (1991) を基にジェネリックメーカーの参入数と1社あたりの期待利潤の関係をモデル化する。

\subsection{潜在的参入者の期待利潤 \label{ep}}
以下の分析では、医薬品の有効成分ごとに市場が画定されることを前提とする。各市場におけるジェネリック医薬品メーカーの参入行動を説明するモデルにおいて、プレイヤーは既存企業である先発品メーカーと潜在的参入者であるジェネリック医薬品メーカーの2種類からなる。潜在的参入者にはAGメーカーを含めないこととする\footnote{現実には、AGと純粋なジェネリック医薬品はほぼ同時期に参入するが、AGの参入の意思決定の主体はAGメーカーではなく、当該市場の先発品メーカーであることを前提としていることによる。}。AGとの競争の下で、個々のジェネリック医薬品メーカーが期待する利潤は次のように表すことができる。
\begin{align}
    \pi (N) = \frac{E_{AG}[V(N) \mid N]}{N} - F
\end{align}
確率変数$N$は参入するジェネリックメーカー数、$V(N)$はジェネリックメーカーが通時的に獲得できる合算レント、$F$は固定費用である。ここでは企業間の同質性を仮定し、1社あたりの合算レントが可変利潤となっている。ジェネリックメーカーが直面する不確実性は、自社を含めジェネリックメーカーが何社参入するのか、先発品メーカーがAGを投入するのか否かの2点であるところ、上式ではAGの有無に関してのみ期待値をとっている。このことは期待値の添え字$AG$に表されている。なお、特許満了後の当該市場において別の先発品の参入は生じないものとする。\par
各医薬品市場のレントは時間を通じて変化する。ここではレントが一定の割合で縮小していくものと仮定する。
\begin{align}
    V(N) = \sum_{t = 1}^{\infty} \rho^{t-1}v_1(N) = \beta v_1(N)
\end{align}
ここで、$\rho$はレントの縮小割合であり$(0<\rho<1)$、$\beta = 1/(1-\rho)$である。上の合算レントをAGが参入しなかった場合$(AG =0)$のそれとする。さらに、限界費用が生産数量に関して一定であると仮定すると、静学的な($t = 1$での)可変利潤$v_1$は下のように表される。
\begin{align}
    v_1(N) = PCM_1(N) \times R_1(N)
\end{align}
$PCM$はプライス・コスト・マージンであり、$R$は収入(売上)である。前者は価格と限界費用の間のマージンを価格で除した値として定義され、主に市場の競争度に依存する変数である。後者は価格と数量の積として定義され、主に市場規模によって規定される。\par
ここで、AGの参入は合算レントの減少をもたらす(Reiffen and Ward, 2007)。そのため、$AG=1$の下では
\begin{align}
    V(N) = \beta \left(v_1(N) - \phi(N) \right) = \beta \frac{v_1(N) - \phi(N)}{v_1(N)} v_1 = \beta d v_1(N)
\end{align}
となる。$\phi$はAGによるレント獲得分を表し、ジェネリックのレントシェア率$d \equiv (v_1 - \phi)/v_1$は$N$によらず一定であると仮定する。$d$ は本来AGとジェネリックの競争によって内生的に決定されるが、この分析では品質や信頼性といったAGの競争優位性をとらえるデータを取得することができなかったため、やむを得ずこの仮定をおく。この点は本研究の課題の1つでもある。ここで、AGが参入する確率を$p \equiv \Pr(AG = 1)$とすると、
\begin{align}
    E_{AG}[V(N) \mid N] = (1-p)\beta v_1(N) + p \beta d v_1(N) = \beta D(p) v_1(N)
\end{align}
ただし、関数$D(p) \equiv 1 - (1-d)p$ はAGの参入期待による期待利潤の減少率を表す。ここで利潤関数は次のように変形できる。
\begin{align}
    \label{pi}
    \pi(N) = \frac{\beta D(p) v_1(N)}{N} - F = \beta \left[ \frac{D(p)v_1(N)}{N} - F' \right]
\end{align}
ただし、$F' \equiv F / \beta$である。

\subsection{AG参入に関する信念の形成}
ジェネリックメーカーは各市場にAGが投入されるか否かを事前に観察することはできない。そこで、ジェネリックメーカーは$t =0$時点で既に特許満了済みの市場を観察することで、$t =1$時点で特許が満了する各市場におけるAG投入の可能性に関して信念を形成するものと考える。\par
ジェネリックメーカーが観察する市場は2つの時点間で異なる。時点$t=0$で既に特許が満了している市場の集合と、 $t=1$時点で初めて参入が可能になる市場の集合をそれぞれ$M_0$と $M_1$によって表し($M_0 \cap M_1=\emptyset$)、それぞれの構成要素である個別市場を$m_t$と表す。\par
各市場における有効成分の特徴とAGの有無の関係は次のように定式化される。
\begin{align}
    \label{belief_est}
    \Pr(AG_{m_0} = 1) = \Pr \left( g(X_{m_0}; \gamma) > 0 \right) 
\end{align}
$X_{m_0}$は市場(有効成分)の属性、$AG_{m_0}$は市場$m_0$でAGの参入が観察された場合に1をとるダミー変数である。$g(X_{m_0};\gamma )$は市場$m_0$にAGを投入することで先発品メーカーが得るネットの利得を表す関数である。各ジェネリックメーカーは$M_0$に所属する市場のデータに基づき、数式(\ref{belief_est})のパラメータ$\gamma$を推定するものとする。そのうえで、各ジェネリックメーカーは推定値$\hat{\gamma}$を使って$M_1$に含まれる各市場においてAGが投入されることに関する主観的確率(信念)$\hat{p}$を形成する。
\begin{align}
    \hat{p}_{m_1} = \Pr \left( g(X_{m_1}; \hat{\gamma}) > 0 \right)
\end{align}
参入の意思決定に際しては、期待利潤の形成に関わる関数$D(p)$が$D(\hat{p})$によって置き換えられる。

\subsection{推定}
各ジェネリックメーカーは、\ref{ep}節で定義した期待利潤の正負によって参入に関する意思を決定する。企業間の同質性を仮定した場合の利潤は参入メーカー数の関数として表されるため、Bresnahan and Reiss (1991)に倣って、均衡状態における参入企業数$N^*$は次の方程式の陰関数として表される。
\begin{align}
    \Pr(N = N^*) = \Pr \left[ \pi(N^{*} + 1) \leq 0 < \pi(N^{*}) \right]
\end{align}
ここで(\ref{pi})より、
\begin{align}
    \pi &(N^{*}) > 0 \nonumber \\
    & \ \Leftrightarrow \frac{Dv_1}{N^{*}} > F' \nonumber \\
    & \ \Leftrightarrow \ln D + \ln v_1  - \ln N^{*} > \ln F' \nonumber \\
    & \ \Leftrightarrow \ln D(p) + \ln v(N^*) - \ln N^{*} + f > \varepsilon
\end{align}
%ここで、$Z$は需要や競争度に関わる変数であり, $Y$はと供給に関わる変数である。ここでは$v_1 \equiv v(N, Z)$, $f(Y) \equiv -F'+ \varepsilon$とした\footnote{変形する際に$\beta$や$D(p) v_1/N^{*}$, $F'$がそれぞれ正であることを確認する必要がある。}。
ここでは$f \equiv -\ln F'$とした\footnote{変形する際に$\beta$や$D(p) v_1/N^{*}$, $F'$がそれぞれ正であることを確認する必要がある。}。$\varepsilon$は固定費用や可変利潤に関わる観察不能な効果であり、標準正規分布に従うものと仮定する。$\pi(N^{*}+1) \geq 0$ についても同様の変形を行うことで、均衡参入数 $N^*$が実現する確率が次のように表現できる。
\begin{align}
    \label{prob}
    \Pr &(N = N^*) \nonumber \\
    &= \Pr \left[ \ln D(p) + \ln v(N^{*}+1) - \ln N^{*} + f \leq \varepsilon < \ln D(p) + \ln v(N^{*}) - \ln N^{*} + f \right]
\end{align}
(\ref{prob})の確率表現に基づいて尤度関数を構築すれば、順序プロビットモデルでパラメータを推定することができる。

\section{データ}
\subsection{データソース}
分析には厚生労働省が公表している薬価基準収載品リストおよびNDBオープンデータに加えて、日本医薬情報センター (JAPIC) が発行しているJAPIC医療用医薬品集を用いた。\par
薬価基準収載品リストは、医療機関等で保険診療に用いられる医療用医薬品として官報に告示されている(薬価基準に収載されている)品目を示したものである。このデータセットには各医薬品の薬価収載コード、医薬品名、主な有効成分とその含有量、薬価、剤型、先発品と後発品の区別などが収められている。\par
NDBオープンデータはレセプト情報や特定健診等情報を収めたデータセットであるが、本分析では前者のみを用いる。レセプトとは、保険診療を行った医療機関が、患者一人一人の診療報酬(医療費)を、 審査支払機関を経由して保険者に請求を行う際の明細書である。このデータセットには、医薬品の薬価収載コードやレセプト数量などがカバーされている\footnote{厚生労働省, 「レセプト情報・特定健診等情報データの第三者提供の在り方に関する報告書」(\url{https://www.mhlw.go.jp/stf/shingi/indexshingi.html})}。\par
JAPIC医療用医薬品集には医療用医薬品のYJコード、医薬品名、一般名、薬価、薬効、剤型、含有量、薬価収載日などが収められている。YJコードとは、薬価収載品のうち統一名収載された個々の品目を区別して付与されるコードであり、銘柄別収載される医薬品に関してはYJコードと薬価収載コードは同一である。統一名収載方式で収載される医薬品は、日本薬局方収載医薬品(局方品)、生物学的製剤基準収載医薬品の一部(ワクチン・血液製剤など)、生薬の一部、および一般名収載品目などが該当するが\footnote{DATA iNDEX , 情報医療ナレッジ (\url{https://www.data-index.co.jp/knowledge/143/})}、これらはサンプルから除外した。一般名とは、メーカーが個別で決められる医薬品名とは異なり、有効成分の組成ごとに決められる名前である。\par
以上のソースから得られたデータセットを薬価収載コードおよびYJコードによって紐づけることでサンプルを構築した。

\subsection{サンプル}
分析における観察単位は市場であり、市場は有効成分ごとに画定する。ジェネリックメーカーの参入行動($t = 1$ における行動)の分析にあたっては、2017年から2019年の間に特許が満了した有効成分を対象(サンプル)とした。サンプルに含める有効成分は、先発品の特許が満了していることが要件となるが、今回取得したデータセットには特許満了時期の情報が含まれない。そこで、個々の有効成分について特許満了後であるか否かを推測した。具体的には、既に特許が満了したことが分かっている有効成分について、先発品とジェネリック医薬品の薬価収載日のラグを算出し、その情報を基に、データセットに含まれる各有効成分の特許満了時期を推測した。特許満了後の医薬品については、最初に収載された先発品と最初に収載されたジェネリック医薬品の収載日のラグは平均で約6年であった。そこで、期初の2017年の6年前にあたる2011年以前に先発品が薬価収載された有効成分のうち、2017年時点でジェネリック医薬品が参入していなかった有効成分をサンプルとして定義した。これらはいずれも2017年までには特許が満了していたと推測される医薬品である。\par
サンプルをこのように構築することで、AGの参入阻止効果に関する推定結果にバイアスがもたらされる可能性はある。なぜなら、最初に収載されたジェネリック医薬品の収載日は、先発品の特許満了のタイミングと必ずしも一致せず、ジェネリック医薬品の収載日以前に特許が満了している場合も考えられるからである。特に、AGが参入インセンティブを低下させるという本分析のシナリオの下では、AGが投入される可能性が高い市場において実際に参入のインセンティブが低下していたためにジェネリックの薬価収載が遅れている可能性があり、このようなサンプルに基づいてAGの参入実績の効果を推定すると、AGが参入企業数に与える負の効果が0の方向にバイアスされることが考えられる。一方で、このような0の方向にバイアスされた効果の推定値に統計的な有意性が認められれば、効果の程度を正確に把握できなくとも、その存在は示されることになる。この意味で、薬価収載日のラグによって選定したサンプルに基づく分析には一定の意義がある。\par
なお、初期 ($t = 0$) にあたるサンプルについては、2014年以前に薬価収載された医薬品のうち、2015年において先発医薬品と後発医薬品がともに収載されていた有効成分に限定して構築した。これはジェネリック医薬品の開発に約3〜5年かかることと2013年に初めてAGが参入したことを踏まえた選定方法である。

\subsection{特定化}
本分析では、過去のAGの投入実績がジェネリック医薬品の参入に与える効果を順序プロビットモデルに基づいて推定する。前節でまとめたように、アウトカムはジェネリック医薬品の参入数であり、分析対象期間の最終年である2019年時点で薬価収載されていた純粋なジェネリック医薬品の数として計測される。\par
AGの上市によるレントの変化率を表す$D$については次の3通りの特定化を行う。
\begin{align}
    \ln D_m = 
    \left\{
        \begin{array}{lc}
            \alpha_1 pastAG_m  &  \\
            \alpha_1 pastAG\_mean_m &  \\
            \alpha_1 belief
        \end{array}
    \right.    
\end{align}
$pastAG_m$市場$m$の先発品メーカーが過去にAGを上市した有効成分単位の品目数の合計であり, $pastAG\_mean_m$は先発品メーカーが過去にAGを上市した有効成分の既存企業あたりの平均品目数である\footnote{後記の記述統計で確認できるように、既存の先発品メーカー数は複数存在する場合がある。}。$belief_m$はジェネリックメーカーが持つAGの参入確率に関する信念である。この値は当該市場の先発品メーカーによる過去のAG投入とその市場の特徴の関係を2項プロビットモデルで推定したのち\footnote{このプロビットモデルの推定結果は補論に示す。}、推定したパラメータを使って当期の市場データからAG参入確率を予測した値である。\par
前述の通り、レント$v_m$はプライス・コスト・マージンと収入の積で表される。本分析ではレントを次式のとおり、競争度合いに関連する既存先発品メーカー数、レセプト売上、代替品の既存先発品メーカー数および市場規模に関連するレセプト売上などの関数として特定化する。
\begin{align}
    \ln v_m = 
    & \  \lambda_1 N\_incumbent_m + \lambda_2 \ln revenue_m+ \lambda_3 \ln revenue\_m \times N\_incumbent_m \nonumber \\
&+\lambda_4 \ln revenue\_hos_m+ \lambda_5 \ln revenue\_hos_m \times N\_incumbent_m \\ &+\lambda_6subst\_inc_m + \lambda_7 \ln subst\_rev_m+ \lambda_8 \ln subst\_rev_m \times subst\_inc_m \nonumber
\end{align}
ここで、$N\_incumbent_m$は先発品メーカー数、$revenue_m$は先発品の金額ベースの売上の合計を表しており、2014年時点での当該有効成分の医薬品の薬価とレセプト数量を掛け合わせて合計したものである。$revenue\_hos_m$は先発品のレセプト売上の内、入院患者に対して処方された医薬品の売上を合計したものであり、Scott Morton (1999)やIizuka(2007)と同様にコントロール変数に加える。$subst\_inc_m$は当該医薬品と薬効を同じくする医薬品の先発品メーカー数であり、$subst\_rev_m$はその売上の合計($revenue_m$と同様に定義)である。\par
固定費用に関連する$f_m$は次のように特定化する。
\begin{align}
    f_m = 
    & \  \eta_1 capsule_m+ \eta_2 tablet_m+ \eta_3 granule_m+ \eta_4 syrup_m+ \eta_5 liquid_m \nonumber \\ 
    & + \eta_6 form\_variety_m+ \eta_7 inactive\_variety_m
\end{align}
$capsule_m$, $tablet_m$, $granule_m$, $syrup_m$, $liquid_m$ はそれぞれ、先発品の剤型にカプセル剤、錠剤、顆粒、シロップ、液が含まれる場合に1をとるダミー変数である。$form\_variety_m$は先発品の剤型の種類数であり、こちらは小分類でカウントしている。$inactive\_variety_m$は先発品に含まれている添加物の種類のカウントである。各変数の基本統計量を表\ref{summary}に示す。
\begin{table}[!htbp] 
    \centering 
    \caption{記述統計} 
    \label{summary} 
    \begin{tabular}{@{\extracolsep{5pt}}lccccccc} 
        \\[-1.8ex]\hline 
        \hline \\[-1.8ex] 
        Statistic & \multicolumn{1}{c}{N} & \multicolumn{1}{c}{Mean} & \multicolumn{1}{c}{St. Dev.} & \multicolumn{1}{c}{Min} & \multicolumn{1}{c}{Pctl(25)} & \multicolumn{1}{c}{Pctl(75)} & \multicolumn{1}{c}{Max} \\ 
        \hline \\[-1.8ex] 
        N\_entry & 204 & 0.627 & 2.481 & 0 & 0 & 0 & 18 \\ 
        belief & 204 & 0.0001 & 0.001 & 0 & 0 & 0 & 0 \\ 
        pastAG & 204 & 1.015 & 1.580 & 0 & 0 & 1.2 & 5 \\ 
        pastAG\_mean & 204 & 0.959 & 1.523 & 0 & 0 & 1.1 & 5 \\ 
        pastAG\_dummy & 204 & 0.363 & 0.482 & 0 & 0 & 1 & 1 \\ 
        ln\_revenue\_total & 204 & 17.695 & 4.639 & 9.125 & 13.665 & 21.777 & 25.230 \\ 
        ln\_revenue\_total\_hos & 204 & 15.263 & 3.908 & 8.029 & 11.816 & 18.512 & 22.357 \\ 
        subst\_act & 204 & 2.157 & 3.568 & 0 & 0 & 3 & 16 \\ 
        ln\_subst\_rev & 204 & 15.449 & 10.193 & 0 & 0 & 23.5 & 27 \\ 
        ln\_subst\_rev\_hos & 204 & 13.384 & 8.940 & 0 & 0 & 20.6 & 25 \\ 
        capsule & 204 & 0.186 & 0.390 & 0 & 0 & 0 & 1 \\ 
        tablet & 204 & 0.711 & 0.455 & 0 & 0 & 1 & 1 \\ 
        granule & 204 & 0.103 & 0.305 & 0 & 0 & 0 & 1 \\ 
        syrup & 204 & 0.034 & 0.182 & 0 & 0 & 0 & 1 \\ 
        liquid & 204 & 0.078 & 0.270 & 0 & 0 & 0 & 1 \\ 
        form\_variety & 204 & 1.221 & 0.530 & 1 & 1 & 1 & 4 \\ 
        inactive\_variety & 204 & 9.279 & 4.466 & 0 & 6 & 11 & 28 \\ 
        \hline \\[-1.8ex] 
    \end{tabular} 
\end{table}

\section{推定結果}
推定結果は表\ref{estimate}にまとめたとおりである\footnote{本稿では順序プロビットモデルでの推定結果のみを示しているが、頑健性チェックのために、Iizuka(2009)にならって負二項回帰モデル (negative binomial) での推定もおこなった。推定結果は概ね順序プロビットモデルのそれと同様であった。}。表\ref{estimate}の左2列はAG変数として$pastAG_m$を含めたモデル、中2列は$pastAG\_mean_m$を含めたモデル、右2列は$belief_m$を含めたモデルである。それぞれの右側の列は説明変数に$\ln revenue\_hos_m$を加えたときの推定結果である。\par
AG変数のパラメータ推定値についてはいずれも統計的に有意な結果は得られなかった。したがって、過去のAGの参入実績がジェネリックの参入インセンティブを低下させるという仮説は採択できない。一方,奇数列における$\ln revenue_m$のパラメータ推定値は10\%水準で統計的に有意であった。偶数列では$\ln revenue_m$は統計的に有意ではないが、$\ln revenue_m$と相関が高い$\ln revenue\_hos_m$が説明変数に含まれていることを考慮すると、推定値が安定していないだけであると考えられる。これらを踏まえると、市場規模と参入ジェネリック数の間には正の関係があることが示唆され、Scott Morton(1999)やIizuka(2009)と同様の結果が得られている。
\begin{table}[!htbp] 
    \centering 
    \caption{推定結果} 
    \label{estimate} 
    %\begin{adjustbox}{width=\columnwidth,center}
    \scalebox{0.8}{
        \begin{tabular}{@{\extracolsep{5pt}}lcccccc} 
        \\[-1.8ex]\hline 
        \hline \\[-1.8ex] 
         & \multicolumn{6}{c}{\textit{Dependent variable:}} \\ 
        \cline{2-7} 
        \\[-1.8ex] & \multicolumn{6}{c}{N\_entry} \\ 
        \\[-1.8ex] & (1) & (2) & (3) & (4) & (5) & (6)\\ 
        \hline \\[-1.8ex] 
         pastAG & 0.040 & 0.038 &  &  &  &  \\ 
          & (0.097) & (0.098) &  &  &  &  \\ 
          %& & & & & & \\ 
         pastAG\_mean &  &  & $-$0.0002 & $-$0.00008 &  &  \\ 
          &  &  & (0.105) & (0.106) &  &  \\ 
          %& & & & & & \\ 
         belief &  &  &  &  & $-$194.33 & $-$190.97 \\ 
          &  &  &  &  & (3,967.7) & (3,967.7) \\ 
          %& & & & & & \\ 
         ln\_revenue & 0.240$^{*}$ & 0.298 & 0.236$^{*}$ & 0.307 & 0.234$^{*}$ & 0.308 \\ 
          & (0.114) & (0.255) & (0.114) & (0.256) & (0.114) & (0.255) \\ 
          %& & & & & & \\ 
         ln\_revenue\_hos &  & $-$0.082 &  & $-$0.101 &  & $-$0.104 \\ 
          &  & (0.321) &  & (0.321) &  & (0.322) \\ 
          %& & & & & & \\ 
         N\_incumbent & 2.764 & 2.509 & 2.705 & 2.419 & 2.678 & 2.391 \\ 
          & (1.837) & (1.983) & (1.841) & (1.983) & (1.837) & (1.975) \\ 
          %& & & & & & \\ 
         capsule & 0.691 & 0.677 & 0.681 & 0.677 & 0.690 & 0.689 \\ 
          & (0.838) & (0.837) & (0.833) & (0.834) & (0.830) & (0.834) \\ 
          %& & & & & & \\ 
         tablet & 0.784 & 0.785 & 0.787 & 0.796 & 0.784 & 0.795 \\ 
          & (0.819) & (0.812) & (0.814) & (0.809) & (0.811) & (0.808) \\ 
          %& & & & & & \\ 
         granule & $-$0.241 & $-$0.279 & $-$0.250 & $-$0.251 &  & $-$0.290 \\ 
          & (0.664) & (0.674) & (0.669) & (0.679) & (0.667) & (0.677) \\ 
          %& & & & & & \\ 
         syrup & 0.509 & 0.479 & 0.478 & 0.446 & 0.473 & 0.440 \\ 
          & (0.819) & (0.827) & (0.821) & (0.828) & (0.817) & (0.824) \\ 
          %& & & & & & \\ 
         liquid & $-$4.966 & $-$5.289 & $-$4.982 & $-$5.776 &  & $-$6.171 \\ 
          & (307.137) & (295.643) & (305.13) & (295.69) & (246.46) & (245.28) \\ 
          %& & & & & & \\ 
         inactive\_variety & $-$0.040 & $-$0.035 & $-$0.040 & $-$0.034 & $-$0.040 & $-$0.035 \\ 
          & (0.043) & (0.044) & (0.043) & (0.044) & (0.043) & (0.044) \\ 
          %& & & & & & \\ 
         subst\_inc & $-$0.814 & $-$0.824 & $-$0.828 & $-$0.843 & $-$0.829 & $-$0.847 \\ 
          & (0.759) & (0.774) & (0.755) & (0.771) & (0.752) & (0.041) \\ 
          %& & & & & & \\ 
         ln\_subst\_rev & 0.008 & 0.008 & 0.009 & 0.009 & 0.009 & 0.009 \\ 
          & (0.017) & (0.017) & (0.017) & (0.017) & (0.017) & (0.017) \\ 
          %& & & & & & \\ 
         subst\_inc:ln\_subst\_rev & 0.033 & 0.033 & 0.034 & 0.034 & 0.034 & 0.034 \\ 
          & (0.029) & (0.030) & (0.035) & (0.030) & (0.029) & (0.030) \\ 
          %& & & & & & \\ 
         ln\_revenue:N\_incumbent & $-$0.110 & $-$0.183 & $-$0.108 & $-$0.191 & $-$0.107 & $-$0.191 \\ 
          & (0.088) & (0.227) & (0.088) & (0.227) & (0.088) & (0.227) \\ 
          %& & & & & & \\ 
         N\_incumbent:ln\_revenue\_hos &  & 0.102 &  & 0.167 &  & 0.118 \\ 
          &  & (0.293) &  & (0.293) &  & (0.292) \\ 
          %& & & & & & \\ 
        \hline \\[-1.8ex] 
        Observations & 204 & 204 & 204 & 204 & 204 & 204 \\ 
        Log Likelihood & $-$92.738 & $-$92.625 & $-$92.822 & $-$92.700 & $-$92.762 & $-$92.646 \\ 
        McFadden's R$^2$ & 0.974 & 0.974 & 0.974 & 0.974 & 0.974 & 0.974 \\ 
        \hline 
        \hline \\[-1.8ex] 
        \textit{Note:} 下段括弧内は標準誤差をあらわす。  & \multicolumn{6}{r}{$^{*}$p$<$0.1; $^{**}$p$<$0.05; $^{***}$p$<$0.01} \\ 
    \end{tabular} 
    %\end{adjustbox}
    }
\end{table}

\section{結論}
本研究では日本の医療用医薬品市場を対象に、新薬メーカーによる過去のAG投入実績が当該市場におけるジェネリックメーカーの参入数に与える影響を順序プロビットモデルに基づいて推定した。その結果、過去のAGの投入実績がジェネリックメーカーの参入確率を低下させるという仮説を支持する結果は得られなかった。この推定結果を踏まえると、先発品メーカーがジェネリックメーカーの将来の期待利潤の形成に与える影響を予見し、参入阻止の戦略の一つとしてAGを市場に投入しているとは言えない。\par
他方で、本研究にはいくつかの課題が残されている。一つ目は、サンプルの構築方法が不完全であることだ。サンプル構築方法が原因となり、過去のAGの参入実績や当期におけるAGが参入する可能性の推測が参入企業数に与える効果の推定値が0の方向にバイアスしている可能性がある。二つ目の課題として、先発品メーカーの企業属性を十分にコントロールできていないことが挙げられる。先発品メーカー間に異質性が存在する場合は、それによって推定結果に内生性バイアスが生じている可能性がある。\par
また、本分析では2期間のモデルを想定したが、短期間ではジェネリックメーカーによる信念の形成が十分に行われていない可能性がある。この点に関しては、より長いスパンのパネルデータを用い、ジェネリックメーカーが信念を逐次的に更新するようなモデルに基づいて分析すれば改善できるだろう。\par


\newpage
\bibliographystyle{jplain}
\addcontentsline{toc}{section}{参考文献}
\begin{thebibliography}{3}
\bibitem{1}
阿部隆徳 (2021), 「バイオAGがバイオ後続品に与える影響の一考察 ー薬価算定における公正取引委員会の知見の有効活用ー」, 公正取引, No. 850, pp.74-84

\bibitem{2}
公正取引委員会, (2015) 「医薬品市場における競争と研究開発インセンティブ」
\bibitem{3}
厚生労働省, (2020)「ロードマップ検証検討事業報告書」
\bibitem{4}
Appelt, Silvia (2010), "Authorized Generic Entry prior to Patent Expiry: Reassessing Incentives for Independent Generic Entry", {\it Munich Discussion Paper}, No. 2010-23
\bibitem{5}
Berndt Ernst, Mortimer Richard, Parece Andrew, (2007), "Do authorized generic drugs deter paragraph IV certifications? Recent evidence", {\it working paper}
\bibitem{6}
Berry, Steven (1992), "Estimation of a Model of Entry in the Airline Industry", {\it Econometrica}, vol. 60, pp. 889-917
\bibitem{7}
Berry, Steven and Reiss, Peter (2007), "Empirical Models of Entry and Market Structure", {\it Handbook of Industrial Organization}, vol. 3, Chapter 29
\bibitem{8}
Bresnahan, Timothy and Reiss, Peter (1991), "Entry and Competition in Concentrated Markets", {\it Journal of Political Economy}, 99 (5), 977–1009.
\bibitem{9}
Federal Trade Commission (2011), "Authorized Generic Drugs: Short-Term Effects and Long-Term Impact"
\bibitem{10}
Iizuka, Toshiaki (2009), "Generic Entry in Regulated Pharmaceutical Market", {\it Japanese Economic Review}, vol. 60, pp.63-81
\bibitem{11}
Kim, Sung-Hwan (2009), "Predatory Reputation in US Airline Markets", {\it International Journal of Industrial Organization}, v.27, pp.592-604
\bibitem{12}
Peelish, Natalie (2020), "Antitrust and Authorized Generics: A New Predation Analysis", {\it Stanford Law Review}, vol.72, pp.791-839
\bibitem{13}
Reiffen, David and Ward, Michael R. (2005), "Generic drug industry dynamics", {\it Review of Economics and Statistics}, 87, pp.37-49
\bibitem{14}
Reiffen, David and Ward, Michael R. (2007), "‘Branded Generic’ as a Strategy to Limit Cannibalization of Pharmaceutical Markets", {\it Managerial and Decision Economics}, 28, pp.251-265
\bibitem{15}
Scott Morton, Fiona (1999), "Entry Decisions in the Generic Pharmaceutical Industry", {\it The RAND Journal of Economics}, vol 30., pp.421-440

\end{thebibliography}

\newpage

\section*{補論}
ジェネリックメーカーの信念の形成における2項プロビットモデルの推定結果は下表の通り。説明変数には既存先発品メーカーのダミー変数を含めている。
\begin{table}[!htbp] 
    \centering 
    \caption{2項プロビットモデルの推定結果} 
    \label{} 
    \scalebox{0.8}{
    \begin{tabular}{@{\extracolsep{5pt}}lc} 
        \\[-1.8ex]\hline 
        \hline \\[-1.8ex] 
         & \multicolumn{1}{c}{\textit{Dependent variable:}} \\ 
        \cline{2-2} 
        \\[-1.8ex] & AG\_entry \\ 
        \hline \\[-1.8ex] 
         ln\_revenue\_total & 8.020 \\ 
          & (1,024.837) \\ 
          %& \\ 
         price & $-$0.0001 \\ 
          & (0.495) \\ 
          %& \\ 
         capsule & $-$15.003 \\ 
          & (310,581.000) \\ 
          %& \\ 
         tablet & $-$2.672 \\ 
          & (14,496.490) \\ 
          %& \\ 
         granule & $-$9.282 \\ 
          & (17,515.160) \\ 
          %& \\ 
         syrup & 14.332 \\ 
          & (31,324.650) \\ 
          %& \\ 
         liquid & $-$15.947 \\ 
          & (2,808.083) \\ 
          %& \\ 
         form\_variety & 11.205 \\ 
          & (1,722.065) \\ 
          %& \\ 
         inactive\_variety & $-$1.092 \\ 
          & (135.964) \\ 
          %& \\ 
         subst\_act & 4.016 \\ 
          & (1,214.762) \\ 
          %& \\ 
         subst\_inc & $-$59.125 \\ 
          & (7,524.987) \\ 
          %& \\ 
         ln\_subst\_rev & 0.554 \\ 
          & (123.582) \\ 
          %& \\ 
         ln\_revenue\_total:N\_incumbent & 0.013 \\ 
          & (710.240) \\ 
          %& \\ 
         subst\_inc:ln\_subst\_rev & 2.129 \\ 
          & (238.943) \\ 
          %& \\ 
         Constant & $-$244.468 \\ 
          & (33,290.960) \\ 
          %& \\ 
        \hline \\[-1.8ex] 
        Observations & 1,127 \\ 
        Log Likelihood & $-$0.00000 \\ 
        Akaike Inf. Crit. & 176.000 \\ 
        \hline 
        \hline \\[-1.8ex] 
        \textit{Note:} 下段括弧内は標準誤差を示している。 & \multicolumn{1}{r}{$^{*}$p$<$0.1; $^{**}$p$<$0.05; $^{***}$p$<$0.01} \\ 
    \end{tabular} 
    }
\end{table} 


\end{document}